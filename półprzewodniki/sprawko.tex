\documentclass[11pt,a4paper]{article}

\usepackage[utf8]{inputenc}
\usepackage{polski}
\usepackage{graphicx}
\usepackage{graphics}
\usepackage{subfig}
\usepackage{float}
\usepackage{geometry}
\usepackage{amsmath}
\usepackage{color}
\usepackage{enumerate}
\usepackage{indentfirst}
\usepackage{hyperref}
\usepackage{multirow}
\usepackage{array}
\usepackage{url}
\usepackage{longtable} 
\newgeometry{lmargin=2.5cm,rmargin=2.5cm}

\title{}
\author{}
\date{}

\begin{document}
\begin{table}[]
\hspace{-2em}
\begin{tabular}{|c|l|c|c|c|c|}
\hline
	\begin{tabular}[c]{@{}c@{}}Wydział\\ FiIS\end{tabular}&
	\multicolumn{2}{l|}{\begin{tabular}[c]{@{}l@{}}Imię i nazwisko\\ 1. Piotr Kowalczyk\\ 2. Marcin Polok\end{tabular}}&
	\begin{tabular}[c]{@{}c@{}}Rok\\ IV\end{tabular}&
	\begin{tabular}[c]{@{}c@{}}Grupa \\ 2\end{tabular}&
	\begin{tabular}[c]{@{}c@{}}Zespół \\ 2\end{tabular}
\\ \hline
	\begin{tabular}[c]{@{}c@{}}LABORATORIUM\\ DETEKCJI\\ PROMIENIOWANIA\end{tabular}&
	\multicolumn{5}{l|}{\textbf{\begin{tabular}[c]{@{}l@{}}Temat\\ Badanie licznika półprzewodnikowego \end{tabular}}}
\\ \hline
	\begin{tabular}[c]{@{}c@{}}Data wykonania\\ 3.11.2016\end{tabular}&
	\multicolumn{1}{c|}{\begin{tabular}[c]{@{}c@{}}Data oddania\\ 30.11.2016 \end{tabular}}&
	\multicolumn{1}{c|}{\begin{tabular}[c]{@{}c@{}}Zwrot do popr.\\~ \end{tabular}}&
	\multicolumn{1}{c|}{\begin{tabular}[c]{@{}c@{}}Data oddania\\~ \end{tabular}}&
	\multicolumn{1}{c|}{\begin{tabular}[c]{@{}c@{}}Data zaliczenia\\~ \end{tabular}}&
	\multicolumn{1}{c|}{\begin{tabular}[c]{@{}c@{}}OCENA\\~ \end{tabular}}
\\ \hline
\end{tabular}
\end{table}


%%%%%%%%%%%%%%%%%%%%%%%%%%%%%%%%%%%%%%%%%%%%%%%%%%%%%%%%%%%%%%%%%%%%%%%%%%%%%%%%%%%%%%%%%%%%%%%%%%%%%%%%%%%%%%%%%%%%%%%%%%%%%%
%                                                CEL ĆWICZENIA                                                               %
%%%%%%%%%%%%%%%%%%%%%%%%%%%%%%%%%%%%%%%%%%%%%%%%%%%%%%%%%%%%%%%%%%%%%%%%%%%%%%%%%%%%%%%%%%%%%%%%%%%%%%%%%%%%%%%%%%%%%%%%%%%%%%
\section{Wstęp teoretyczny}
%%%%%%%%%%%%%%%%%%%%%%%%%%%%%%%%%%%%%%%%%%%%%%%%%%%%%%%%%%%%%%%%%%%%%%%%%%%%%%%%%%%%%%%%%%%%%%%%%%%%%%%%%%%%%%%%%%%%%%%%%%%%%%
%                                                PRZEBIEG ĆWICZENIA                                                          %
%%%%%%%%%%%%%%%%%%%%%%%%%%%%%%%%%%%%%%%%%%%%%%%%%%%%%%%%%%%%%%%%%%%%%%%%%%%%%%%%%%%%%%%%%%%%%%%%%%%%%%%%%%%%%%%%%%%%%%%%%%%%%%
\section{Przebieg ćwiczenia}
%%%%%%%%%%%%%%%%%%%%%%%%%%%%%%%%%%%%%%%%%%%%%%%%%%%%%%%%%%%%%%%%%%%%%%%%%%%%%%%%%%%%%%%%%%%%%%%%%%%%%%%%%%%%%%%%%%%%%%%%%%%%%%
%                                                WYNIKI                                                                      %
%%%%%%%%%%%%%%%%%%%%%%%%%%%%%%%%%%%%%%%%%%%%%%%%%%%%%%%%%%%%%%%%%%%%%%%%%%%%%%%%%%%%%%%%%%%%%%%%%%%%%%%%%%%%%%%%%%%%%%%%%%%%%%
\section{Wyniki}

\begin{figure}[H]
\centering
%\resizebox{.82\linewidth}{!}{\input{czas_martwy1.tex}}
\caption{Pomiar czasu martwego obu detektorów.}
\label{dead_time}
\end{figure}

\section{Wnioski}

\section{Dane pomiarowe}

\begin{longtable}{cc}
\caption{////}\\
\label{dt2}
U[V]	&	t[ms] \\ \hline
\endhead
1600	&0,124 \\
\end{longtable}

\begin{thebibliography}{9}
\bibitem{skrypt1}
Skrypt Ćwiczenia laboratoryjne z jądrowych metod pomiarowych dostępny pod adresem:\\http://winntbg.bg.agh.edu.pl/skrypty3/0364/dziunikowski-kalita.pdf 
\end{thebibliography}
\end{document}
